%%%%%%%%%%%%%%%%%%%%%%%%%%%%%%%%%%%%%%%%%
% fphw Assignment
% LaTeX Template
% Version 1.0 (27/04/2019)
%
% This template originates from:
% https://www.LaTeXTemplates.com
%
% Authors:
% Class by Felipe Portales-Oliva (f.portales.oliva@gmail.com) with template
% content and modifications by Vel (vel@LaTeXTemplates.com)
%
% Template (this file) License:
% CC BY-NC-SA 3.0 (http://creativecommons.org/licenses/by-nc-sa/3.0/)
%
%%%%%%%%%%%%%%%%%%%%%%%%%%%%%%%%%%%%%%%%%

%----------------------------------------------------------------------------------------
%	PACKAGES AND OTHER DOCUMENT CONFIGURATIONS
%----------------------------------------------------------------------------------------

\documentclass[
	12pt, % Default font size, values between 10pt-12pt are allowed
	%letterpaper, % Uncomment for US letter paper size
	%spanish, % Uncomment for Spanish
]{fphw}

% Template-specific packages
\usepackage[utf8]{inputenc} % Required for inputting international characters
\usepackage[T1]{fontenc} % Output font encoding for international characters
\usepackage{mathpazo} % Use the Palatino font
\usepackage{graphicx} % Required for including images
\usepackage{booktabs} % Required for better horizontal rules in tables
\usepackage{listings} % Required for insertion of code
\usepackage{enumerate} % To modify the enumerate environment
\usepackage{color} % To color the code
\usepackage{float} % here for H placement parameter
\definecolor{mygreen}{rgb}{0,0.6,0}
\definecolor{mygray}{rgb}{0.5,0.5,0.5}
\definecolor{mymauve}{rgb}{0.58,0,0.82}

%----------------------------------------------------------------------------------------
%	ASSIGNMENT INFORMATION
%----------------------------------------------------------------------------------------

\title{Homework \#1} % Assignment title
\author{Andrew Hankins} % Student name
\date{February 8th, 2023} % Due date
\institute{University of Alabama \\ College of Engineering} % Institute or school name
\class{Artificial Intelligence (CS 565)} % Course or class name
\professor{Dr. Monica Anderson Herzog} % Professor or teacher in charge of the assignment

%----------------------------------------------------------------------------------------

\begin{document}

\maketitle % Output the assignment title, created automatically using the information in the custom commands above

%----------------------------------------------------------------------------------------
%	ASSIGNMENT CONTENT
%----------------------------------------------------------------------------------------

%-------------------------------------------------------------------
% SECTION 1: UPDATED PYTHON FILES
%-------------------------------------------------------------------

\section{Updated Python Files}

\begin{problem}
	\lstinputlisting[
		title=achankins.py, % Caption above the listing
		label=lst:achankins1, % Label for referencing this listing
		language=Python, % Use Perl functions/syntax highlighting
		frame=single, % Frame around the code listing
		showstringspaces=false, % Don't put marks in string spaces
		numbers=left, % Line numbers on left
		numberstyle=\tiny, % Line numbers styling
		breaklines=true,
		columns=fullflexible,
		firstline=1,
		lastline= 65,
		commentstyle=\color{mygreen},
		keywordstyle=\color{blue},
		numberstyle=\tiny\color{mygray},
		basicstyle=\tiny
	]{../src/achankins.py}
\end{problem}

\begin{problem}
	\lstinputlisting[
		title=compare\_all\_moves\_strategy.py, % Caption above the listing
		label=lst:achankins2, % Label for referencing this listing
		language=Python, % Use Perl functions/syntax highlighting
		frame=single, % Frame around the code listing
		showstringspaces=false, % Don't put marks in string spaces
		numbers=left, % Line numbers on left
		numberstyle=\tiny, % Line numbers styling
		breaklines=true,
		columns=fullflexible,
		firstline=1,
		lastline= 95,
		commentstyle=\color{mygreen},
		keywordstyle=\color{blue},
		numberstyle=\tiny\color{mygray},
		basicstyle=\tiny
	]{../src/compare_all_moves_strategy.py}
\end{problem}

\begin{problem}
	\lstinputlisting[
		title=compare\_all\_moves\_strategy.py, % Caption above the listing
		label=lst:comparemoves, % Label for referencing this listing
		language=Python, % Use Perl functions/syntax highlighting
		frame=single, % Frame around the code listing
		showstringspaces=false, % Don't put marks in string spaces
		numbers=left, % Line numbers on left
		numberstyle=\tiny, % Line numbers styling
		breaklines=true,
		columns=fullflexible,
		firstline=96,
		lastline= 195,
		commentstyle=\color{mygreen},
		keywordstyle=\color{blue},
		numberstyle=\tiny\color{mygray},
		basicstyle=\tiny,
		firstnumber=96
	]{../src/compare_all_moves_strategy.py}
\end{problem}

%----------------------------------------------------------------------
% SECTION 2: EXPLANATION OF NOVEL FEATURE
%----------------------------------------------------------------------

\section{Explanation of Novel Feature}

%-------------------------------------------------------------------------------------------
% SECTION 3: COMPARISON OF 5 BEST WEIGHTING FUNCTIONS
%-------------------------------------------------------------------------------------------

\section{Comparison of 5 Best Weighting Functions}

\quad\quad The following section will discuss the five best weighting functions that I tested throughout my searching process.

%----------------------------------------------------
% Subsection 1: Best Weighting Function

\subsection{Best Weighting Function}

\hfill\\ \\  The first weighting function that I found

\begin{lstlisting}[
		label=lst:weighting1, % Label for referencing this listing
		language=Python, % Use Perl functions/syntax highlighting
		frame=single, % Frame around the code listing
		showstringspaces=false, % Don't put marks in string spaces
		numbers=left, % Line numbers on left
		numberstyle=\tiny, % Line numbers styling
		breaklines=true,
		columns=fullflexible,
		commentstyle=\color{mygreen},
		keywordstyle=\color{blue},
		numberstyle=\tiny\color{mygray},
		basicstyle=\tiny,
	]
class player1_achankins(CompareAllMoves):

    # Function that will evaluate the board
    def evaluate_board(self, myboard, colour):
    board_stats = self.assess_board(colour, myboard)

    # Attempt to normalize the features between a value of 0...1 and weight them
    board_value =  0.75 * (board_stats['sum_distances'] / 163.0)                                        + \
                            -0.75 * (board_stats['number_of_singles'] / 7.0)                                     + \
                            -0.75 * (board_stats['number_occupied_spaces'] / 7.0)                          + \
                            -0.25 * (board_stats['opponents_taken_pieces'] / 1.0)                            + \
                             0.9   * (board_stats['sum_distances_to_endzone'] / 75.0)                      + \
                             0.9   * (board_stats['sum_single_distance_away_from_home'] / 100.0) + \
                             1.0   * (board_stats['pieces_on_board'] / 15.0)                                        + \
                            -1.0   * (board_stats['sum_distances_opponent'] / 163.0)
    return board_value

\end{lstlisting}

\begin{table}[H]
	\centering
	\begin{tabular}{||c | c c c c c||}
		\hline
		Opponent & Run 1&   Run 2 & Run 3 & Avg. Win Rate & Std. Dev. \\ [0.5ex]
		\hline\hline
		CAMWD &  6 & 10 & 15 & 100\% & 1 \\
		\hline
		MFBS & 2 & 3 & 4 & 100\% & 1 \\ [1ex]
		\hline
	\end{tabular}
	\caption{Weighting algorithm 1}
	\label{table:1}
\end{table}

%---------------------------------------------------------------
% Subsection 2: Second Best Weighting Function

\subsection{Second Best Weighting Function}

\hfill\\ \\  The second weighting function that I found

\begin{lstlisting}[
		label=lst:weighting2, % Label for referencing this listing
		language=Python, % Use Perl functions/syntax highlighting
		frame=single, % Frame around the code listing
		showstringspaces=false, % Don't put marks in string spaces
		numbers=left, % Line numbers on left
		numberstyle=\tiny, % Line numbers styling
		breaklines=true,
		columns=fullflexible,
		commentstyle=\color{mygreen},
		keywordstyle=\color{blue},
		numberstyle=\tiny\color{mygray},
		basicstyle=\tiny,
	]
class player1_achankins(CompareAllMoves):

    # Function that will evaluate the board
    def evaluate_board(self, myboard, colour):
    board_stats = self.assess_board(colour, myboard)

    # Attempt to normalize the features between a value of 0...1 and weight them
    board_value =  0.75 * (board_stats['sum_distances'] / 163.0)                                        + \
                            -0.75 * (board_stats['number_of_singles'] / 7.0)                                     + \
                            -0.75 * (board_stats['number_occupied_spaces'] / 7.0)                          + \
                            -0.25 * (board_stats['opponents_taken_pieces'] / 1.0)                            + \
                             0.9   * (board_stats['sum_distances_to_endzone'] / 75.0)                      + \
                             0.9   * (board_stats['sum_single_distance_away_from_home'] / 100.0) + \
                             1.0   * (board_stats['pieces_on_board'] / 15.0)                                        + \
                            -1.0   * (board_stats['sum_distances_opponent'] / 163.0)
    return board_value

\end{lstlisting}

\begin{table}[ht]
	\centering
	\begin{tabular}{||c | c c c c c||}
		\hline
		Opponent & Run 1&   Run 2 & Run 3 & Avg. Win Rate & Std Dev. \\ [0.5ex]
		\hline\hline
		CAMWD &  6 & 10 & 15 & 100\% & 1 \\
		\hline
		MFBS & 2 & 3 & 4 & 100\% & 1 \\ [1ex]
		\hline
	\end{tabular}
	\caption{Weighting algorithm 2}
	\label{table:2}
\end{table}

%---------------------------------------------------------------
% Subsection 3: Third Best Weighting Function

\subsection{Third Best Weighting Function}

\hfill\\ \\  The third weighting function that I found

\begin{lstlisting}[
		label=lst:weighting3, % Label for referencing this listing
		language=Python, % Use Perl functions/syntax highlighting
		frame=single, % Frame around the code listing
		showstringspaces=false, % Don't put marks in string spaces
		numbers=left, % Line numbers on left
		numberstyle=\tiny, % Line numbers styling
		breaklines=true,
		columns=fullflexible,
		commentstyle=\color{mygreen},
		keywordstyle=\color{blue},
		numberstyle=\tiny\color{mygray},
		basicstyle=\tiny,
	]
class player1_achankins(CompareAllMoves):

    # Function that will evaluate the board
    def evaluate_board(self, myboard, colour):
    board_stats = self.assess_board(colour, myboard)

    # Attempt to normalize the features between a value of 0...1 and weight them
    board_value =  0.75 * (board_stats['sum_distances'] / 163.0)                                        + \
                            -0.75 * (board_stats['number_of_singles'] / 7.0)                                     + \
                            -0.75 * (board_stats['number_occupied_spaces'] / 7.0)                          + \
                            -0.25 * (board_stats['opponents_taken_pieces'] / 1.0)                            + \
                             0.9   * (board_stats['sum_distances_to_endzone'] / 75.0)                      + \
                             0.9   * (board_stats['sum_single_distance_away_from_home'] / 100.0) + \
                             1.0   * (board_stats['pieces_on_board'] / 15.0)                                        + \
                            -1.0   * (board_stats['sum_distances_opponent'] / 163.0)
    return board_value
\end{lstlisting}

\begin{table}[ht]
	\centering
	\begin{tabular}{||c | c c c c c||}
		\hline
		Opponent & Run 1&   Run 2 & Run 3 & Avg. Win Rate & Std Dev. \\ [0.5ex]
		\hline\hline
		CAMWD &  6 & 10 & 15 & 100\% & 1 \\
		\hline
		MFBS & 2 & 3 & 4 & 100\% & 1 \\ [1ex]
		\hline
	\end{tabular}
	\caption{Weighting algorithm 3}
	\label{table:3}
\end{table}

%---------------------------------------------------------------
% Subsection 4: Fourth Best Weighting Function

\subsection{Fourth Best Weighting Function}

\hfill\\ \\  The fourth weighting function that I found

\begin{lstlisting}[
		label=lst:weighting4, % Label for referencing this listing
		language=Python, % Use Perl functions/syntax highlighting
		frame=single, % Frame around the code listing
		showstringspaces=false, % Don't put marks in string spaces
		numbers=left, % Line numbers on left
		numberstyle=\tiny, % Line numbers styling
		breaklines=true,
		columns=fullflexible,
		commentstyle=\color{mygreen},
		keywordstyle=\color{blue},
		numberstyle=\tiny\color{mygray},
		basicstyle=\tiny,
	]
class player1_achankins(CompareAllMoves):

    # Function that will evaluate the board
    def evaluate_board(self, myboard, colour):
    board_stats = self.assess_board(colour, myboard)

    # Attempt to normalize the features between a value of 0...1 and weight them
    board_value =  0.75 * (board_stats['sum_distances'] / 163.0)                                        + \
                            -0.75 * (board_stats['number_of_singles'] / 7.0)                                     + \
                            -0.75 * (board_stats['number_occupied_spaces'] / 7.0)                          + \
                            -0.25 * (board_stats['opponents_taken_pieces'] / 1.0)                            + \
                             0.9   * (board_stats['sum_distances_to_endzone'] / 75.0)                      + \
                             0.9   * (board_stats['sum_single_distance_away_from_home'] / 100.0) + \
                             1.0   * (board_stats['pieces_on_board'] / 15.0)                                        + \
                            -1.0   * (board_stats['sum_distances_opponent'] / 163.0)
    return board_value
\end{lstlisting}

\begin{table}[ht]
	\centering
	\begin{tabular}{||c | c c c c c||}
		\hline
		Opponent & Run 1&   Run 2 & Run 3 & Avg. Win Rate & Std Dev. \\ [0.5ex]
		\hline\hline
		CAMWD &  6 & 10 & 15 & 100\% & 1 \\
		\hline
		MFBS & 2 & 3 & 4 & 100\% & 1 \\ [1ex]
		\hline
	\end{tabular}
	\caption{Weighting algorithm 4}
	\label{table:4}
\end{table}

%---------------------------------------------------------------
% Subsection 5: Fifth Best Weighting Function

\subsection{Fifth Best Weighting Function}

\hfill\\ \\  The fifth weighting function that I found

\begin{lstlisting}[
		label=lst:weighting5, % Label for referencing this listing
		language=Python, % Use Perl functions/syntax highlighting
		frame=single, % Frame around the code listing
		showstringspaces=false, % Don't put marks in string spaces
		numbers=left, % Line numbers on left
		numberstyle=\tiny, % Line numbers styling
		breaklines=true,
		columns=fullflexible,
		commentstyle=\color{mygreen},
		keywordstyle=\color{blue},
		numberstyle=\tiny\color{mygray},
		basicstyle=\tiny,
	]
class player1_achankins(CompareAllMoves):

    # Function that will evaluate the board
    def evaluate_board(self, myboard, colour):
    board_stats = self.assess_board(colour, myboard)

    # Attempt to normalize the features between a value of 0...1 and weight them
    board_value =  0.75 * (board_stats['sum_distances'] / 163.0)                                        + \
                            -0.75 * (board_stats['number_of_singles'] / 7.0)                                     + \
                            -0.75 * (board_stats['number_occupied_spaces'] / 7.0)                          + \
                            -0.25 * (board_stats['opponents_taken_pieces'] / 1.0)                            + \
                             0.9   * (board_stats['sum_distances_to_endzone'] / 75.0)                      + \
                             0.9   * (board_stats['sum_single_distance_away_from_home'] / 100.0) + \
                             1.0   * (board_stats['pieces_on_board'] / 15.0)                                        + \
                            -1.0   * (board_stats['sum_distances_opponent'] / 163.0)
    return board_value
\end{lstlisting}

\begin{table}[ht]
	\centering
	\begin{tabular}{||c | c c c c c||}
		\hline
		Opponent & Run 1&   Run 2 & Run 3 & Avg. Win Rate & Std. Dev. \\ [0.5ex]
		\hline\hline
		CAMWD &  6 & 10 & 15 & 100\% & 1 \\
		\hline
		MFBS & 2 & 3 & 4 & 100\% & 1 \\ [1ex]
		\hline
	\end{tabular}
	\caption{Weighting algorithm 5}
	\label{table:5}
\end{table}

%-------------------------------------------------------------------
% SECTION 4: PLAYER COMPARISONS
%-------------------------------------------------------------------

\section*{Player Comparisons}

\quad \quad The following section compares the performance of both player1\_achankins and player2\_achankins against the MoveFurthestBackStrategy and CompareAllMovesWeightingDistance players. This test was done by running three sets of 200 games per player per opponent. After each run, the winning percentage of the player was recorded. Once all three runs had been completed, the average win percentage and standard deviation was calculated. This testing system was designed to accurately assess the newly created players by using a sufficient amount of games multiple times in order to ensure the result was correct.

\begin{table}[ht]
	\centering
	\begin{tabular}{||c | c c c c c||}
		\hline
		Player & Run 1&   Run 2 & Run 3 & Avg. Win Rate & Std. Dev. \\ [0.5ex]
		\hline\hline
		player1\_achankins &  96.0\% & 94.0\% & 94.5\% & 94.83\% & 0.85 \\
		\hline
		player2\_achankins & 2 & 3 & 4 & 100\% & 1 \\ [1ex]
		\hline
	\end{tabular}
	\caption{Comparison against MoveFurthestBackStrategy}
	\label{table:6}
\end{table}

\begin{table}[ht]
	\centering
	\begin{tabular}{||c | c c c c c||}
		\hline
		Player & Run 1 &   Run 2 & Run 3  & Avg. Win Rate & Std. Dev. \\ [0.5ex]
		\hline\hline
		Player1\_achankins &  61.5\% & 61.5\% & 65.5\% & 62.83\% & 1.89 \\
		\hline
		player2\_achankins & 2 & 3 & 4 & 100\% & 1 \\ [1ex]
		\hline
	\end{tabular}
	\caption{Comparison against CompareAllMovesWeightingDistance}
	\label{table:7}
\end{table}

%-------------------------------------------------------------------
% SECTION 6: MINIMAX GAME TREE
%-------------------------------------------------------------------

\pagebreak

\section*{Game Tree}

\quad \quad The minimax algorithm is a strategy designed to select the optimal move in an adversarial game by assuming the opponent always selects the move that
will minimize your score. By looking ahead we are able to see the worst case scenario from each roll and select the best possible course of action. For example from the
below minimax tree we can determine that going to state 10 will give us the best possible outcome.

\lstinputlisting[language={}, basicstyle=\tiny]{../minimax_tree.txt}

\end{document}