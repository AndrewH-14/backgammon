%%%%%%%%%%%%%%%%%%%%%%%%%%%%%%%%%%%%%%%%%
% fphw Assignment
% LaTeX Template
% Version 1.0 (27/04/2019)
%
% This template originates from:
% https://www.LaTeXTemplates.com
%
% Authors:
% Class by Felipe Portales-Oliva (f.portales.oliva@gmail.com) with template
% content and modifications by Vel (vel@LaTeXTemplates.com)
%
% Template (this file) License:
% CC BY-NC-SA 3.0 (http://creativecommons.org/licenses/by-nc-sa/3.0/)
%
%%%%%%%%%%%%%%%%%%%%%%%%%%%%%%%%%%%%%%%%%

%----------------------------------------------------------------------------------------
%	PACKAGES AND OTHER DOCUMENT CONFIGURATIONS
%----------------------------------------------------------------------------------------

\documentclass[
	12pt, % Default font size, values between 10pt-12pt are allowed
	%letterpaper, % Uncomment for US letter paper size
	%spanish, % Uncomment for Spanish
]{fphw}

% Template-specific packages
\usepackage[utf8]{inputenc} % Required for inputting international characters
\usepackage[T1]{fontenc} % Output font encoding for international characters
\usepackage{mathpazo} % Use the Palatino font

\usepackage{graphicx} % Required for including images

\usepackage{booktabs} % Required for better horizontal rules in tables

\usepackage{listings} % Required for insertion of code

\usepackage{enumerate} % To modify the enumerate environment

\usepackage{color} % To color the code

\usepackage{float} % here for H placement parameter

\definecolor{mygreen}{rgb}{0,0.6,0}
\definecolor{mygray}{rgb}{0.5,0.5,0.5}
\definecolor{mymauve}{rgb}{0.58,0,0.82}

%----------------------------------------------------------------------------------------
%	ASSIGNMENT INFORMATION
%----------------------------------------------------------------------------------------

\title{Homework \#1} % Assignment title

\author{Andrew Hankins} % Student name

\date{February 8th, 2023} % Due date

\institute{University of Alabama \\ College of Engineering} % Institute or school name

\class{Artificial Intelligence (CS 565)} % Course or class name

\professor{Dr. Monica Anderson Herzog} % Professor or teacher in charge of the assignment

%----------------------------------------------------------------------------------------

\begin{document}

\maketitle % Output the assignment title, created automatically using the information in the custom commands above

%----------------------------------------------------------------------------------------
%	ASSIGNMENT CONTENT
%----------------------------------------------------------------------------------------

%------------------------------------------------

\section{Updated Python Files}

\begin{problem}
	\lstinputlisting[
		title=achankins.py, % Caption above the listing
		label=lst:achankins1, % Label for referencing this listing
		language=Python, % Use Perl functions/syntax highlighting
		frame=single, % Frame around the code listing
		showstringspaces=false, % Don't put marks in string spaces
		numbers=left, % Line numbers on left
		numberstyle=\tiny, % Line numbers styling
		breaklines=true,
		columns=fullflexible,
		firstline=1,
		lastline= 65,
		commentstyle=\color{mygreen},
		keywordstyle=\color{blue},
		numberstyle=\tiny\color{mygray},
		basicstyle=\tiny
	]{../src/achankins.py}
\end{problem}

\begin{problem}
	\lstinputlisting[
		title=compare\_all\_moves\_strategy.py, % Caption above the listing
		label=lst:achankins2, % Label for referencing this listing
		language=Python, % Use Perl functions/syntax highlighting
		frame=single, % Frame around the code listing
		showstringspaces=false, % Don't put marks in string spaces
		numbers=left, % Line numbers on left
		numberstyle=\tiny, % Line numbers styling
		breaklines=true,
		columns=fullflexible,
		firstline=1,
		lastline= 95,
		commentstyle=\color{mygreen},
		keywordstyle=\color{blue},
		numberstyle=\tiny\color{mygray},
		basicstyle=\tiny
	]{../src/compare_all_moves_strategy.py}
\end{problem}

\begin{problem}
	\lstinputlisting[
		title=compare\_all\_moves\_strategy.py, % Caption above the listing
		label=lst:achankins3, % Label for referencing this listing
		language=Python, % Use Perl functions/syntax highlighting
		frame=single, % Frame around the code listing
		showstringspaces=false, % Don't put marks in string spaces
		numbers=left, % Line numbers on left
		numberstyle=\tiny, % Line numbers styling
		breaklines=true,
		columns=fullflexible,
		firstline=96,
		lastline= 195,
		commentstyle=\color{mygreen},
		keywordstyle=\color{blue},
		numberstyle=\tiny\color{mygray},
		basicstyle=\tiny,
		firstnumber=96
	]{../src/compare_all_moves_strategy.py}
\end{problem}

%----------------------------------------------------------------------------------------

\section*{Explanation of Novel Feature}

%----------------------------------------------------------------------------------------

\section*{Comparison of 5 Best Weighting Functions}

The five best weighting functions that I found.

\subsection*{Best Weighting Function}

\hfill\\ \\  The first weighting function that I found

\begin{lstlisting}[
		title=compare\_all\_moves\_strategy.py, % Caption above the listing
		label=lst:achankins10, % Label for referencing this listing
		language=Python, % Use Perl functions/syntax highlighting
		frame=single, % Frame around the code listing
		showstringspaces=false, % Don't put marks in string spaces
		numbers=left, % Line numbers on left
		numberstyle=\tiny, % Line numbers styling
		breaklines=true,
		columns=fullflexible,
		commentstyle=\color{mygreen},
		keywordstyle=\color{blue},
		numberstyle=\tiny\color{mygray},
		basicstyle=\tiny,
	]
class player1_achankins(CompareAllMoves):

    # Function that will evaluate the board
    def evaluate_board(self, myboard, colour):
    board_stats = self.assess_board(colour, myboard)

    # Attempt to normalize the features between a value of 0...1 and weight them
    board_value =  0.75 * (board_stats['sum_distances'] / 163.0)                                        + \
                            -0.75 * (board_stats['number_of_singles'] / 7.0)                                     + \
                            -0.75 * (board_stats['number_occupied_spaces'] / 7.0)                          + \
                            -0.25 * (board_stats['opponents_taken_pieces'] / 1.0)                            + \
                             0.9   * (board_stats['sum_distances_to_endzone'] / 75.0)                      + \
                             0.9   * (board_stats['sum_single_distance_away_from_home'] / 100.0) + \
                             1.0   * (board_stats['pieces_on_board'] / 15.0)                                        + \
                            -1.0   * (board_stats['sum_distances_opponent'] / 163.0)
    return board_value

\end{lstlisting}

\begin{table}[H]
	\centering
	\begin{tabular}{||c | c c c c c||}
		\hline
		Opponent & Run 1&   Run 2 & Run 3 & Avg. Win Rate & Std. Dev. \\ [0.5ex]
		\hline\hline
		MoveFurthestBackStrategy &  6 & 10 & 15 & 100\% & 1 \\
		\hline
		CompareAllWeightingDistance & 2 & 3 & 4 & 100\% & 1 \\ [1ex]
		\hline
	\end{tabular}
	\caption{Weighting algorithm 1}
	\label{table:1}
\end{table}

\subsection*{Second Best Weighting Function}

\hfill\\ \\  The first weighting function that I found

\begin{lstlisting}[
		title=compare\_all\_moves\_strategy.py, % Caption above the listing
		label=lst:achankins10, % Label for referencing this listing
		language=Python, % Use Perl functions/syntax highlighting
		frame=single, % Frame around the code listing
		showstringspaces=false, % Don't put marks in string spaces
		numbers=left, % Line numbers on left
		numberstyle=\tiny, % Line numbers styling
		breaklines=true,
		columns=fullflexible,
		commentstyle=\color{mygreen},
		keywordstyle=\color{blue},
		numberstyle=\tiny\color{mygray},
		basicstyle=\tiny,
	]
class player1_achankins(CompareAllMoves):

    # Function that will evaluate the board
    def evaluate_board(self, myboard, colour):
    board_stats = self.assess_board(colour, myboard)

    # Attempt to normalize the features between a value of 0...1 and weight them
    board_value =  0.75 * (board_stats['sum_distances'] / 163.0)                                        + \
                            -0.75 * (board_stats['number_of_singles'] / 7.0)                                     + \
                            -0.75 * (board_stats['number_occupied_spaces'] / 7.0)                          + \
                            -0.25 * (board_stats['opponents_taken_pieces'] / 1.0)                            + \
                             0.9   * (board_stats['sum_distances_to_endzone'] / 75.0)                      + \
                             0.9   * (board_stats['sum_single_distance_away_from_home'] / 100.0) + \
                             1.0   * (board_stats['pieces_on_board'] / 15.0)                                        + \
                            -1.0   * (board_stats['sum_distances_opponent'] / 163.0)
    return board_value

\end{lstlisting}

\begin{table}[ht]
	\centering
	\begin{tabular}{||c | c c c c c||}
		\hline
		player & Run 1&   Run 2 & Run 3 & Avg. Win Rate & Std Dev. \\ [0.5ex]
		\hline\hline
		player1\_achankins &  6 & 10 & 15 & 100\% & 1 \\
		\hline
		player2\_achankins & 2 & 3 & 4 & 100\% & 1 \\ [1ex]
		\hline
	\end{tabular}
	\caption{Weighting algorithm 2}
	\label{table:2}
\end{table}

\subsection*{Third Best Weighting Function}

\hfill\\ \\  The first weighting function that I found

\begin{lstlisting}[
		title=compare\_all\_moves\_strategy.py, % Caption above the listing
		label=lst:achankins10, % Label for referencing this listing
		language=Python, % Use Perl functions/syntax highlighting
		frame=single, % Frame around the code listing
		showstringspaces=false, % Don't put marks in string spaces
		numbers=left, % Line numbers on left
		numberstyle=\tiny, % Line numbers styling
		breaklines=true,
		columns=fullflexible,
		commentstyle=\color{mygreen},
		keywordstyle=\color{blue},
		numberstyle=\tiny\color{mygray},
		basicstyle=\tiny,
	]
class player1_achankins(CompareAllMoves):

    # Function that will evaluate the board
    def evaluate_board(self, myboard, colour):
    board_stats = self.assess_board(colour, myboard)

    # Attempt to normalize the features between a value of 0...1 and weight them
    board_value =  0.75 * (board_stats['sum_distances'] / 163.0)                                        + \
                            -0.75 * (board_stats['number_of_singles'] / 7.0)                                     + \
                            -0.75 * (board_stats['number_occupied_spaces'] / 7.0)                          + \
                            -0.25 * (board_stats['opponents_taken_pieces'] / 1.0)                            + \
                             0.9   * (board_stats['sum_distances_to_endzone'] / 75.0)                      + \
                             0.9   * (board_stats['sum_single_distance_away_from_home'] / 100.0) + \
                             1.0   * (board_stats['pieces_on_board'] / 15.0)                                        + \
                            -1.0   * (board_stats['sum_distances_opponent'] / 163.0)
    return board_value

\end{lstlisting}

\begin{table}[ht]
	\centering
	\begin{tabular}{||c | c c c c c||}
		\hline
		player & Run 1&   Run 2 & Run 3 & Avg. Win Rate & Std Dev. \\ [0.5ex]
		\hline\hline
		player1\_achankins &  6 & 10 & 15 & 100\% & 1 \\
		\hline
		player2\_achankins & 2 & 3 & 4 & 100\% & 1 \\ [1ex]
		\hline
	\end{tabular}
	\caption{Weighting algorithm 3}
	\label{table:3}
\end{table}

\subsection*{Fourth Best Weighting Function}

\hfill\\ \\  The first weighting function that I found

\begin{lstlisting}[
		title=compare\_all\_moves\_strategy.py, % Caption above the listing
		label=lst:achankins10, % Label for referencing this listing
		language=Python, % Use Perl functions/syntax highlighting
		frame=single, % Frame around the code listing
		showstringspaces=false, % Don't put marks in string spaces
		numbers=left, % Line numbers on left
		numberstyle=\tiny, % Line numbers styling
		breaklines=true,
		columns=fullflexible,
		commentstyle=\color{mygreen},
		keywordstyle=\color{blue},
		numberstyle=\tiny\color{mygray},
		basicstyle=\tiny,
	]
class player1_achankins(CompareAllMoves):

    # Function that will evaluate the board
    def evaluate_board(self, myboard, colour):
    board_stats = self.assess_board(colour, myboard)

    # Attempt to normalize the features between a value of 0...1 and weight them
    board_value =  0.75 * (board_stats['sum_distances'] / 163.0)                                        + \
                            -0.75 * (board_stats['number_of_singles'] / 7.0)                                     + \
                            -0.75 * (board_stats['number_occupied_spaces'] / 7.0)                          + \
                            -0.25 * (board_stats['opponents_taken_pieces'] / 1.0)                            + \
                             0.9   * (board_stats['sum_distances_to_endzone'] / 75.0)                      + \
                             0.9   * (board_stats['sum_single_distance_away_from_home'] / 100.0) + \
                             1.0   * (board_stats['pieces_on_board'] / 15.0)                                        + \
                            -1.0   * (board_stats['sum_distances_opponent'] / 163.0)
    return board_value

\end{lstlisting}

\begin{table}[ht]
	\centering
	\begin{tabular}{||c | c c c c c||}
		\hline
		player & Run 1&   Run 2 & Run 3 & Avg. Win Rate & Std Dev. \\ [0.5ex]
		\hline\hline
		player1\_achankins &  6 & 10 & 15 & 100\% & 1 \\
		\hline
		player2\_achankins & 2 & 3 & 4 & 100\% & 1 \\ [1ex]
		\hline
	\end{tabular}
	\caption{Weighting algorithm 4}
	\label{table:4}
\end{table}

\pagebreak

\subsection*{Fifth Best Weighting Function}

\hfill\\ \\  The first weighting function that I found

\begin{lstlisting}[
		title=compare\_all\_moves\_strategy.py, % Caption above the listing
		label=lst:achankins10, % Label for referencing this listing
		language=Python, % Use Perl functions/syntax highlighting
		frame=single, % Frame around the code listing
		showstringspaces=false, % Don't put marks in string spaces
		numbers=left, % Line numbers on left
		numberstyle=\tiny, % Line numbers styling
		breaklines=true,
		columns=fullflexible,
		commentstyle=\color{mygreen},
		keywordstyle=\color{blue},
		numberstyle=\tiny\color{mygray},
		basicstyle=\tiny,
	]
class player1_achankins(CompareAllMoves):

    # Function that will evaluate the board
    def evaluate_board(self, myboard, colour):
    board_stats = self.assess_board(colour, myboard)

    # Attempt to normalize the features between a value of 0...1 and weight them
    board_value =  0.75 * (board_stats['sum_distances'] / 163.0)                                        + \
                            -0.75 * (board_stats['number_of_singles'] / 7.0)                                     + \
                            -0.75 * (board_stats['number_occupied_spaces'] / 7.0)                          + \
                            -0.25 * (board_stats['opponents_taken_pieces'] / 1.0)                            + \
                             0.9   * (board_stats['sum_distances_to_endzone'] / 75.0)                      + \
                             0.9   * (board_stats['sum_single_distance_away_from_home'] / 100.0) + \
                             1.0   * (board_stats['pieces_on_board'] / 15.0)                                        + \
                            -1.0   * (board_stats['sum_distances_opponent'] / 163.0)
    return board_value

\end{lstlisting}

\begin{table}[ht]
	\centering
	\begin{tabular}{||c | c c c c c||}
		\hline
		player & Run 1&   Run 2 & Run 3 & Avg. Win Rate & Std. Dev. \\ [0.5ex]
		\hline\hline
		player1\_achankins &  6 & 10 & 15 & 100\% & 1 \\
		\hline
		player2\_achankins & 2 & 3 & 4 & 100\% & 1 \\ [1ex]
		\hline
	\end{tabular}
	\caption{Weighting algorithm 5}
	\label{table:5}
\end{table}

%----------------------------------------------------------------------------------------

\section*{Player Comparisons}

\begin{table}[ht]
	\centering
	\begin{tabular}{||c | c c c c c||}
		\hline
		player & Run 1&   Run 2 & Run 3 & Avg. Win Rate & Std. Dev. \\ [0.5ex]
		\hline\hline
		player1\_achankins &  6 & 10 & 15 & 100\% & 1 \\
		\hline
		player2\_achankins & 2 & 3 & 4 & 100\% & 1 \\ [1ex]
		\hline
	\end{tabular}
	\caption{Comparison against MoveFurthestBackStrategy}
	\label{table:6}
\end{table}

\begin{table}[ht]
	\centering
	\begin{tabular}{||c | c c c c c||}
		\hline
		player & Run 1&   Run 2 & Run 3 & Avg. Win Rate & Std. Dev. \\ [0.5ex]
		\hline\hline
		player1\_achankins &  6 & 10 & 15 & 100\% & 1 \\
		\hline
		player2\_achankins & 2 & 3 & 4 & 100\% & 1 \\ [1ex]
		\hline
	\end{tabular}
	\caption{Comparison against CompareAllMovesWeightingDistance}
	\label{table:7}
\end{table}

%----------------------------------------------------------------------------------------

\section*{Game Tree}


\end{document}